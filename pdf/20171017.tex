\documentclass{article}
\usepackage{ctex}
\usepackage{amssymb,amsmath} 
\usepackage{bm} 
\usepackage{graphicx}
\usepackage{float}
\usepackage{overpic}
\usepackage[amsmath,thmmarks]{ntheorem}
\usepackage{esvect}
\usepackage{diagm}
\usepackage{epstopdf}
\usepackage{color}
\usepackage[amsmath,thmmarks]{ntheorem}
\theoremstyle{plain}% default
\newtheorem{thm}{定理}[section]
\newtheorem{lem}[thm]{引理}
\newtheorem{prop}[thm]{命题}
\newtheorem{cor}{推论}


\theoremstyle{definition}
\theorembodyfont{\normalfont}
\newtheorem{defn}{定义}
\newtheorem{exmp}{例}
\newtheorem{xca}[exmp]{Exercise}
\theoremstyle{remark}
\newtheorem*{rem}{注}
\newtheorem*{conj}{Borel conjecture}
\newtheorem*{note}{Note}
\newtheorem{case}{Case}

\title{Fundamental Groups of Small Covers}
\date{}
\begin{document}
\maketitle
\section{Introduction}
%%%%%%%%%%%%%%%%%%%%%%%%%%%%%%%%%%%%%%%%%%%%%%%%%%%%%%%%%%%%
\subsection{Small Cover}
%polytope-----------~~~~~~~~~~~
{\em 凸多面体$P$}是指$\mathbb{R}^n$中非空有限多个点集的凸包,或者等价的是$\mathbb{R}^n$中有限个半空间的有界交,
即
$$P=conv\{p_1,p_2,\cdots,p_{\ell}\}=\{x\in \mathbb{R}^n :\langle l_i,x\rangle\geq -a_i,i=1,2,\cdots,m\}$$
其中$l_i$为$(\mathbb{R}^n)^*$中的线性函数,$a_i\in \mathbb{R}$.

凸多面体的{\em 维数}就是指凸包或者有界交的维数。若无特殊说明,本文中的所考虑的多面体均指$\mathbb{R}^n$中的$n$维凸多面体,记为$P$. 我们把$P$的边界记为$K$. 把$P$的内部记为$P^{\circ}$. 凸子集$F\subset P$称为$P$的{\em 面},若$F$是多面体$P$与某一个半空间$V=\{x\in \mathbb{R}^n :\langle l,x\rangle\geq -a\}$的交,且$P^{\circ}\cap \partial V =\varnothing$.子集$\varnothing$和$P$本身都为$P$的面,称为{\em 平凡面};其他的面称为{\em 真面}. $P$的$0$维面称为$P$的{\em 顶点},$P$的$1$维面称为$P$的{\em 边},$P$的$n-1$维面称为$P$的{\em facet}. 记$f_i$为$P$的$i$维面的个数,称$\mathbf{f}(P)=(f_0,f_1,\cdots,f_{n-1})$为$P$的{\em $f$-vector}. 取$f_{-1}=1$,则$P$的{\em $h$-vector} $\mathbf{h}(P)=(h_0,h_1,\cdots,h_n)$由下面等式定义
$$h_0 t^n+\cdots+h_{n-1}t+h_n=(t-1)^n+f_0(t-1)^{n-1}+\cdots+f_{n-1}$$
由Dehn-Sommerville 关系知$h_i=h_{n-i},i=0,1,\cdots,n$,为方便我们在本文中将$P$的facets的个数记为$f_{n-1}=m$,即$P$的facets集为$\mathcal{F}(P)=\{F_1,F_2,\cdots,F_m\}$.

我们称多面体$P$是{\em 单}(simple)的,若$P$的每个顶点恰好是$P$中$n$个facets的交,等价地,每个顶点处恰好有$n$条边.单多面体中任意余维数为$k$的面$f$总可以(唯一)表示为$f=F_1\cap F_2 \cap \cdots \cap F_k$,其中$F_1,F_2,\cdots,F_k$为包含$f$的facets.
%small cover------------


取$\mathbb{Z}_2=\{1,-1\}$为二元乘法群或者模空间,$\mathbb{Z}_2^n$表示它们的乘积,$e_i$表示$\mathbb{Z}_2^n$第$i$个标准向量. 
设$P$为$n$维单凸多面体,$\mathcal{F}(P)$为$P$的facets集,对每一个 facet $F_i\in \mathcal{F}(P)$,我们定义一个染色$\lambda(F_i)\in \mathbb{Z}_2^n$,使得对$P$的每一个顶点$p=F_1\cap F_2 \cap \cdots \cap F_n$,满足$span\{\lambda(F_1),\lambda(F_2),\cdots,\lambda(F_n)\}\cong \mathbb{Z}_2^n$. 进一步我们称$\lambda:\mathcal{F}(P)\longrightarrow \mathbb{Z}_2^n$为下面将要构造的small cover $M$的示性函数. 需要注意的是,下文出现的多面体指数之间的乘法,均默认为群$\mathbb{Z}_2^n$中的乘法运算. 

{\rem 对于任意单凸多面体,满足上面条件的染色不一定存在,参考 Davis-Januszkiewicz \cite{DJ1}. %{\em convex polytope, coxeter orbifolds and torus action }
{\em Nonexample 1.22(Duals of cyclic ploytope)}}

现在我们定义单凸多面体$P$上的small cover. 对任意点$x\in P$,记$f(x)$为$P$中包含$x$为相对内点的唯一的面,例如$x$为$P$内部的点时,则$f(x)=P$; $x$为$P$的顶点时,则$f(x)=F_1\cap F_2 \cap \cdots \cap F_n$,其中$\{F_1,F_2,\cdots,F_n\}$为包含点$x$的$n$个facets. 
不妨设$f(x)=F_1\cap F_2 \cap \cdots \cap F_k$为$P$的任意一个固定的余$k$维面,记$G_{f(x)}=\langle\lambda(F_1),\lambda(F_2),\cdots,\lambda(F_k)\rangle=\langle\lambda(F_i):x\in F_i\rangle$. 则定义{small cover}为
\begin{equation}\label{eq1}
M=(P\times\mathbb{Z}^n_2)/\sim
\end{equation}
$(x,g)\sim (y,h)$当且仅当$x=y,g^{-1}h\in G_{f(x)}$.这里$G_{f(x)}<\mathbb{Z}_2^n$实际上是点$x$处的{\em isotopy subgroup},i.e. $\{g\in \mathbb{Z}_2^n :gx=x\}$. 进一步,设$\pi:M\longrightarrow P$为一个自然的投射.

另外,我们也可以比较直观的构造一个small cover. 任取$P$的一个顶点$p_0$,不妨记$p_0$附近的facets 为$F_1,F_2,\cdots,F_n$,且对应facet上的染色为$\lambda(F_i)=e_i,i=1,2,\cdots,n$. 首先我们把$P$放到$\mathbb{R}^n$的第一卦限中,使得$p_0$与原点重合,第$i$个facet $F_i$落在$x_i=0$的坐标面上. 然后我们将$P$沿着坐标面反射,得到原点附近的$P$的$2^n$个copy,我们把这$2^n$个$P$的copy组成的多面体记为$Q$,$p_0$自然的落在$Q$的内部. 我们给第$g$个坐标卦限的copy一个自然的标号$g\in\mathbb{Z}_2^n$. 最后我们再将$Q$剩余的facets按照染色信息成对粘合起来,具体第$g_1$个$P$的copy的facet ~$F_{i}$与第$g_2$个$P$的facet~$F_{j}$粘,当且仅当$i=j,~g_1^{-1}g_2=\lambda(F_i)$.这样就得到$P$上的small cover $M$.

%设$\pi:M\longrightarrow P$为一个自然的投射. 事实上,将单多面体$P$ 视为orbifold,则 small cover 是一个 right-angle Coxeter orbifold,局部和orbifold $\mathbb{R}^n/\mathbb{Z}_2^n$同构,映射$\pi:M\longrightarrow P$ 是$P$上的一个正则的 orbifold covering,$\mathbb{Z}_2^n$是它的covering transformation group. 参考Davis \cite{DJ1}.

{\prop small cover 为连通闭流形.}\\
{\bf 证明:}参考Davis-Januszkiewicz \cite{DJ1}.{\em 性质 1.7}

%%%%%%%%%%%%%%%%%%%%%%%%%%%%%%%%%%%%%%%%%%%%%%%%%%%%%%%%%%%
\subsection{Examples of Small Covers}
{\exmp
当$P=\triangle^n$时,$\mathcal{F}(P)$上本质上只有一种染色,如$n=2$时,
\begin{figure}[H]
 \centering
 \includegraphics[width=\textwidth]{picture/Tn-1.eps}
  \put(-330,80){ \Large$e_1$}
  \put(-265,80){ \Large$e_2$}
   \put(-300,35){ \Large$e_1e_2$}
\end{figure}
先将$P$在一个点处粘,得到一个大的四边形,由染色信息知它的对边沿着箭头方向粘,这是一个$\mathbb{R}P^2$.
}

{\exmp
当$P^2$为四边形时,$\mathcal{F}(P)$上有下面两种不同的染色,
\begin{figure}[H]
 \centering
 \includegraphics[width=\textwidth]{picture/Sq1.eps}
   \put(-335,95){ \Large$e_1$}
  \put(-240,95){ \Large$e_1$}
    \put(-288,140){ \Large$e_2$}    
    \put(-288,50){ \Large$e_2$}
\end{figure}
\begin{figure}[H]
 \centering
 \includegraphics[width=\textwidth]{picture/Sq2.eps}
    \put(-335,95){ \Large$e_1$}
  \put(-240,95){ \Large$e_1e_2$}
    \put(-288,140){ \Large$e_2$}    
    \put(-288,50){ \Large$e_2$}
\end{figure}
同样的操作,我们可以分别得到$T^2$和 Klein bottle. 
}

{\exmp($P^2$是一个$m$边形时)

$M$是由$4$个 $m$-gon 沿边粘成的曲面,所以$M$的欧拉数为$\chi(M)=4-m$. 当$m$为奇数时,$M$为$m-2$个$\mathbb{R}P^2$的连通和;当$m$为偶数时,$M$为$m-2$个$\mathbb{R}P^2$的连通和或着为$\frac{m-2}{2}$个$T^2$的连通和. 所以small cover 决定了除$S^2$外的所有二维闭曲面.
}

在本文中,我们主要通过构造small cover的一种自然的胞腔分解来计算基本群的群表示. 我们由Hurewicz 定理知道,胞腔复形的基本群可以由它们的二维骨架确定,所以在本文中,我们将构造small cover 的胞腔结构,计算基本群时,仅考虑它们的二维骨架. 另外我们将small cover $M$的$k$维骨架记为$M[k]$,将多面体$P$的$k$维骨架记为$P[k]$.

%%%%%%%%%%%%%%%%%%%%%%%%%%%%%%%%%%%%%%%%

\section{Cell Structure}
%构造一种实用的胞腔结构
由于单凸多面体具有很好的组合性质,所以我们可以通过不同的方式来构造Small cover的胞腔结构. 比如由单多面体面结构诱导的胞腔结构;small cover的perfect 胞腔结构,见Davis-Januszkiewicz \cite{DJ1}.; 由单多面体的 cubical subdivision 所诱导的胞腔结构,见 Buchstaber \cite{Buch}. 在下面一节我们在$Q$上作类似的 cubical subdivision, 来构造small cover 一种更自然的胞腔分解,在这种胞腔结构,可以方便的得到small cover 基本群的一个简洁的表示.
对于一般的单凸多面体$P$,总存在它的一个cubical subdivision,我们将这种分解拉到small cover,我们自然地可以得到small cover 一种胞腔分解. 而$Q$一般来说未必是单的,所以我们下面构造的这种分解不是严格意义的cubical. 进一步,这多面体$Q$的面结构诱导的$M$的一个胞腔分解,我们下面构造的胞腔分解实际上是这种胞腔结构Poincare 意义上的对偶.

%%%%%%%%%%%%%%%%%%%%%%
%youtu
%%%%%%%%%%%%%%%%%%%%%
%在下面我们在$Q^n=P\times \mathbb{Z}_2^n /\sim$中考虑类似的分解,进而给出small cover一种更自然的胞腔分解.
\subsection{Definitions and Constructions}
同上面,我们首先将$|\mathbb{Z}_2^n|=2^n$个多面体$P$的copy在$P$的任一顶点$p_0$处粘合,得到一个大的$n$维多面体$Q$,这里$Q$也可以看作将多面体$P$沿着它的一点$p_0$附近的facets作反射得到.
%,所以对于$M$,局部上$(\mathbb{Z}_2^n不变)$都可由反射构造,染色信息实际上不决定$M$的局部信息.

由$Q$的构造知,$Q$中的每一个$P$自然地拥有一个标号$g\in    \mathbb{Z}_2^n$,我们将第$g$个多面体$P$记为$P_g$,将$Q$中$P_g$的第$i$个facet $F_i$记为$F_{i,g}$. 若$P_g$的$k$维面 $f^k_i\subset \partial Q$,此时$f^k_i$称为$Q$的{\em 外face},否则称为$Q$的{\em 内face},将$Q$的内、外面集分别记为$in(Q),out(Q)$. 接下来把$Q$的外facets按照染色信息配对粘合就可以得到商空间 -- small cover $M$. 我们注意到$M$的所有的facets上存在一种自然的配对结构. 配对的规则由$P$上的染色$\lambda$决定. 这种配对结构有助于我们描述$M$的基本群. 下面,我们引入$M$的面配对结构的定义.

{\defn facets-pair structure of $X$. 
\begin{figure}[H]
 \centering
 \includegraphics[width=0.5\textwidth]{picture/FP1.eps}
\put(-95,68){ $f$}
\put(-80,110){ $F_{i_1,g_1}$}
\put(-55,85){ $F_{i_2,g_1}$}
\put(-128,110){ $F_{j_1,g_2}$}
\put(-150,85){ $F_{i_3,g_2}$}
\put(-128,25){ $F_{i_4,g_3}$}
\put(-150,50){ $F_{i_3,g_3}$}
\put(-80,25){ $F_{j_4,g_4}$}
\put(-55,50){ $F_{j_2,g_4}$}
\end{figure}
设$X$为一个$n$维连通拓扑空间,$X$可以由若干个单凸多面体$\{P_g^n:g=1,2,\cdots,N\}$粘合而成,我们记$P_g$的第$i$个facet $F_i$为$F_{i,g}$,并且满足下面两个条件:\\
1、任意facet $F_{i,g_1}$唯一配对$F_{j,g_2}$%(\neq F_{i,g_1})$,
. 即存在一个同胚$\tau_{i,g_1}:F_{i,g_1}\longrightarrow F_{j,g_2}$与$\tau_{j,g_1}:F_{j,g_2}\longrightarrow F_{i,g_1}$使得$\tau_{i,g_1}=\tau_{j,g_2}^{-1}$. 我们称$\widehat{F}=\{F_{i,g_1},F_{j,g_2}\}$为一个{\em facet对},称$F_{j,g_2}$为$F_{i,g_1}$的{\em 配对facet}.\\%配对面%
2、对任意余二维面$f=F_{i_1,g_1}\cap F_{i_2,g_1}$,如果$\tau_{{i_1},{g_1}}(f)=F_{j_1,g_2}\cap F_{j_3,g_2}$,$\tau_{{i_2},{g_1}}(f)=F_{j_2,g_4}\cap F_{j_4,g_4}$,则$\tau_{{j_3},{g_2}}\tau_{{i_1},{g_1}}(f)=\tau_{{j_4},{g_4}}\tau_{{i_2},{g_1}}(f)=F_{i_3,g_3}\cap F_{i_4,g_3}$. 这里不排除$F_{j_2,g_4}=F_{j_3,g_2}$或者$F_{i_2,g_1}=F_{i_3,g_3}$.

则我们称$\mathcal S=\{\widehat{F}_{i,g},\tau_{i,g}\}$为$\{P_l^n\}$上的一个{\em facets-pairing structure},
$\tau_{i,g}:F_{i,g_1}\longrightarrow F_{j,g_2}$为$\mathcal S$的{\em structure map}.记一步,若$X$为闭的,我们称$\mathcal S$是$M$的一个{\em 完全}的 facets-pairing structure}  $\hfill{} \Box$
%%这里有个图




事实上,$\mathcal{F}(P)$ 上的示性函数$\lambda:\mathcal{F}(P)\longrightarrow \mathbb{Z}_2^n$决定了$M$上的一个配对结构. $F_{i,g_1}\sim F_{j,g_2}$当且仅当$F_i=F_j,~ \lambda(F_i)=(g_1)^{-1}g_2$.反之,若知道$\{P_g^n:g=1,2,\cdots,N\}$上的一个完全配对结构,我们也可以构造出一个闭流形$M$. 
进一步由$\lambda(F_i)=(g_1)^{-1}g_2$得$g_2=g_1\cdot \lambda(F_i)$,即对$Q$的任意一个facet $F_{i,g}$,他的配对facet为$F_{i,\phi_i(g)}$,其中$\phi_i(g)=g\cdot\lambda(F_i):\mathbb{Z}_2^n\longrightarrow \mathbb{Z}_2^n$. 下面,我们把$M$的facets pair 记为$\{F_{i,g},F_{i,\phi_i(g)}\}, \forall g\in\mathbb{Z}_2^n$.
$Q$到$P$有一个自然地投射,我们记为
\begin{equation}\label{eq2}
\bar{\pi}:Q\longrightarrow P
\end{equation}


下面构造$M$的cell structure.
%我们记$M$的$k$维骨架为$M[k]$. 
首先我们将$M[0]$取为点$p_0$,并且设为$M$的基点.
%首先我们取点$p_0$为$M[0]$,并且设为$M$的基点. 
我们在$Q$的每一对余$1$维面处构造$1$-cells.对$Q$的每对facets pair$\{F_{i,g},F_{i,\phi_i(g)}\}$(包括所有的内facets、外facets),任取$F_{i,g}$,$F_{i,\phi_i(g)}$内部的点$a_{i,g},a_{i,\phi_i(g)}$(不妨取为$F_{i,g}$,$F_{i,\phi_i(g)}$的重心),使得$\bar{\pi}(a_{i,g})=\bar{\pi}(a_{i,\phi_i(g)})=a_i\in P$,在$Q$的内部取连接$p_0$到$a_{i,g},a_{i,\phi_i(g)}$的两条简单有向道路(不妨取为直线段),记为$\vv{a_{i,g}},\vv{a_{i,\phi_i(g)}}$. 则$\vv{a_{i,g}}(\vv{a_{i,\phi_i(g)}})^{-1}$为$M$中以$p_0$为起点的一条有向闭路,记为$x_{i,g}$,另外记$x_{i,\phi_i(g)}=x_{i,g}^{-1}$,它表示$M$中以$p_0$为起点的有向闭路$\vv{a_{i,\phi_i(g)}}(\vv{a_{i,g}})^{-1}$.若我们不考虑$x_{i,g}$(或 $x_{i,\phi_i(g)}$)的方向,则$x_{i,g}-\{p_0\}\cong e^1$(或$x_{i,\phi_i(g)}-\{p_0\}\cong e^1$),这里$e^k$表示$M$一个$k$维cell. $M$中的每一对facets pair 都决定了一个$1$-cell. 在上述构造中,所有的$\{x_{i,g}\}$都仅交于$0$-skelton $p_0$ 处. 这样我们就获得$M$的$1$-skelton $M[1]=\bigvee\limits_{p_0} x_{i,g}$. 
\begin{figure}[H]
 \centering
 \includegraphics[width=\textwidth]{picture/Cell-1.eps}
\put(-174,75){$p_0$}
\put(-100,60){\Large$V_2$}
\put(-270,60){\Large$V_1$}
\put(-300,95){$a_{i,g}$}
\put(-300,28){$a_{j,g}$}
\put(-63,104){$a_{i,\phi_i(g)}$}
\put(-63,35){$a_{j,\phi_i(g)}$}
\put(-345,65){$f_1$}
\put(-318,56){$b_1$}
\put(-8,65){$f_2$}
\put(-35,56){$b_2$}
\end{figure}
我们在余$2$维面处构造$2$-cells. 设$f_1=F_{i,g}\cap F_{j,g}$为$Q$的任意一个余$2$维面,则令$f_2=F_{i,\phi_i(g)}\cap F_{j,\phi_i(g)}$,$f_3=F_{i,\phi_i\phi_j(g)}\cap F_{j,\phi_i\phi_j(g)}$,$f_4=F_{i,\phi_j(g)}\cap F_{j,\phi_j(g)}$,使得$\{\bar{\pi}(f_k),k=1,2,3,4\}$ 在$P$中的像相同,记为$f$,这里$\phi_i\phi_j(g)=\phi_i(g\cdot \lambda(f_j))=g\cdot \lambda(f_j)\cdot \lambda(f_i)$. 取$f$内部的一个点$b$,对应$f_k$上的点设为$b_k,~k=1,2,3,4$. 取$V_1$为经过点$b_k,p_0,a_{i,g},a_{j,g}$的二维简单区域,如取$b$为$span\{\vv{a_i},\vv{a_j}\}$与 $f$的交点,其中$\vv{a_i}=\bar{\pi}(\vv{a_{i,g}}), \vv{a_j}=\bar{\pi}(\vv{a_{j,g}})$,这里的$span\{\vv{a_i},\vv{a_j}\}\overset{\Delta}{=}\{\vv{x}=k_1\vv{a_i}+k_2\vv{a_j},~k_1,k_2\geq 0\}$. 则$V_1=span\{\vv{a_{i,g}},\vv{a_{j,g}}\}\cap P_g \cong D_{+}^2$. 
类似确定$V_2=span\{\vv{a_{i,\phi_i(g)}},\vv{a_{j,\phi_i(g)}}\}\cap P_{\phi_i(g)},V_3=span\{\vv{a_{i,\phi_i\phi_j(g)}},\\~\vv{a_{j,\phi_i\phi_j(g)}}\}\cap P_{\phi_i\phi_j(g)},V_4=span\{\vv{a_{i,\phi_j(g)}},\vv{a_{j,\phi_j(g)}}\}\cap P_{\phi_j(g)}$,则$\{V_k:k=1,2,3,4\}$在$M$中实际上粘合成一个闭的$D^2$,记为$D^2_f$,且$D^2_f$的边界落在$M$的$1$-skelton中. 对应的二维cell $e_f^2=(D^2_f)^{\circ}$. 这样就得到$2$-skelton $M[2]=M[1]\cup \{e_f^2\}$.

依次进行下去,我们可以在$Q$的余$k$维面$f_l^k=F_{i_1,g}\cap F_{i_2,g}\cap \cdots \cap F_{i_k,g}$处可构造$M$的$k-cells$. 我们可以类似取$V_l=span\{a_{i_1,g}, a_{i_2,g},\cdots, a_{i_k,g}\}\cap P_g, l=1,2,\cdots,2^k$,它们在$M$中粘成一个$k$维闭圆盘,记为$D^k$,则$\partial D^k$落在$M[k-1]$中,且$D^k$对应$M$的$k$-cell 可以为
\begin{equation*}\label{eq3}
e^k\cong (D^k)^\circ=\left(\bigcup_{\{l=1,2,\cdots,2^k\}} V_l\right)^{\circ}.
\end{equation*}

 最终我们可以在$Q$的顶点处构造$M$的$h_0$个$n-cells$.

{\rem 在上述构造中,若点$p_0\in F_{i,g}$,则$a_{i,g}$可能包含在$F_{i,g}$中,此时我们的构造方法依然适应,且facet $F_{i,g}$对应的$x_{i,g}$在$Q$中与点道路同伦. }

事实上,对于具有facets pair 结构的任意拓扑流形都可类似构造其胞腔结构.如我们考虑
%%这里有个例子和说明
{\exmp 我们将三角形沿着他们对应的边粘和得到一个$S^2$
\begin{figure}[H]
 \centering
 \includegraphics[width=\textwidth]{picture/S2.eps}
\end{figure}
按照上面步骤,我们可以得到$S^2$的一个胞腔分解$S^2=e_0\cup e^1 \cup e_1^2\cup e_2^2$}
%%利用这种胞腔结构计算基本群

\subsection{Calculation and Example}
在这种胞腔结构下,我们可以得到$\pi_1(M)$的一个比较简洁的群表示. 下面我们分析$M$的基本群. small cover 的基本群$\pi_1(M)$的生成元可取为facets对应的有向闭路$\{x_{i,g}\}$.  $\pi_1(M)$的关系由二维胞腔及配对关系决定. 
%已知$F_{i,g_1}\sim F_{i,g_2}$当且仅当$(g_1)^{-1}g_2=\lambda(F_i)$,即$g_2=g_1\cdot \lambda(F_i)=\phi_i(g_1)$,所以
对任意facet pair  $\{F_{i,g},F_{i,\phi_i(g)}\}$对应一对互逆的生成元$x_{i,g}$,$x_{i,\phi_i(g)}$,即配对关系为$x_{i,g}x_{i,\phi_i(g)}=1$.  对于任意余二维面$f=F_{i,g}\cap F_{j,g}(\neq \varnothing)\subset Q$
%(或者等价地$f=F_{i}\cap F_{j}(\neq \varnothing)\subset P$)
,由$f$确定的二维胞腔$e_f$决定一个关系$r_f=\partial D^2_f=x_{i,g}x_{j,\phi_i(g)}x_{i,\phi_i\phi_j(g)}x_{j,\phi_j(g)}=1$,即$x_{i,g}x_{j,\phi_i(g)}=(x_{i,\phi_i\phi_j(g)}x_{j,\phi_j(g)})^{-1}$\\$=(x_{j,\phi_j(g)})^{-1}(x_{i,\phi_i\phi_j(g)})^{-1}=x_{j,g}x_{i,\phi_j(g)}$. 
%%此处有图
从而我们得到$\pi_1(M)$的一个群表示.
%$(P,g)$有个自然的序$l\in \mathbb{Z}^n_2$,我们取基本群生成元为那些$g_1<g_2$的facets 对应的生成元.$Q$中每组余$2$维面对应的$D$确定一个长度为$4$的关系$r(V)=x_{i,g}x_{j,\phi_i(g)}x_{i,\phi_j(g)}^{-1}x_{j,g}^{-1}$.
%\begin{multline}
%\pi_1(M)=\langle x_{i,g},i=1,2,\cdots,m,g\in \mathbb{Z}_2^n:\\
%x_{i,g}x_{j,\phi_i(g)}x_{i,\phi_j(g)}^{-1}x_{j,g}^{-1}=1,\forall f=F_{i,g}\cap F_{j,g}\neq \varnothing\rangle
%\end{multline}
%其中$\phi_i(g)=\lambda(F_i)$
%更进一步,我们将$\{x_{i,g_1}\}$的逆$\{x_{i,g_2}\}$放入基本群的生成系中(或者我们忽略$(P,g)$的序,考虑$Q$的所有facets 对应的生成元),则$r(V)$具有一个漂亮的表达$r(V)=x_{i,g}x_{j,\phi_i(g)}x_{i,\phi_i(g)\phi_j(g)}x_{j,\phi_j(g)}$
\begin{multline}\label{eq0}
\pi_1(M)=\langle x_{i,g},i=1,2,\cdots,m,g\in \mathbb{Z}_2^n:x_{i,g}x_{i,\phi_i(g)}=1,\forall i,g;\\
x_{i,g}x_{j,\phi_i(g)}=x_{j,g}x_{i,\phi_j(g)},\forall f=F_{i,g}\cap F_{j,g}\neq \varnothing\rangle
\end{multline}
其中$\phi_i(g)=g\cdot\lambda(F_i)$


我们记$\mathcal{F}_1(Q)$为$Q$的内facets集;记$\mathcal{F}_2(Q)$为$Q$的内facets集附近的facets集,即$\mathcal{F}_2(Q)=\{F\in \mathcal{F}(Q)\cap \partial{Q}:\exists G \in\mathcal{F}_1(Q), st.~ F\cap G\neq \varnothing \}$;记$\mathcal{F}_3(Q)$为$Q$外facets集剩余的facets集. 则我们有下面结论.
{\lem \label{lem3}
$\forall F_i\in \mathcal{F}(P)$ 固定,则$\bar{\pi}^{-1}(F_i)=\{F_{i,g}:g\in\mathbb{Z}_2^n\}$对应的生成元$\{x_{i,g}:g\in\mathbb{Z}_2^n\}$是彼此相关的. 特别地,当$F_{i,g}\in \mathcal{F}_1(Q)$时,$x_{i,g}=1$; 当$F_{i,g_1}, F_{i,g_2}\in \mathcal{F}_2(Q)$时,若$F_{i,g_1}\cap F_{i,g_2}\neq \varnothing$,则$x_{i,g_1}= x_{i,g_2}$,否则$x_{i,g_1}= (x_{i,g_2})^{-1}$.

进一步,设$f=F_{i,g}\cap F_{j,g}$为$Q$中任意一个固定的余二维面. 当$F_{i,g}$和$F_{j,g}$都属于$\mathcal{F}_1(Q)$,即$f$为内面时,$f$对应的关系为$1$;当$F_{i,g}$和$F_{j,g}$分别属于$\mathcal{F}_2(Q)$和$\mathcal{F}_1(Q)$时,$f$对应的关系为$x_{i,g}=x_{i,\phi_j(g)}$.
}\\
{\bf 证明:}
若$F_{i,g}$为内facets,则$\vv{x_{i,g}}$包含在$Q$的内部,可缩为点道路,故$x_{i,g}=1$. 
对于内余2维面 $f=F_{i,g}\cap F_{j,g}$确定的关系,为内生成元的组合,故也是平凡的. 
若$F_{i,g},F_{j,g}$分别为外面和内面,不妨设$F_{i,g}$为外面,$F_{j,g}$为内面,则$x_{j,g}=x_{j,\phi_i(g)}=1$,所以$f$对应的关系为$x_{i,g}=x_{i,\phi_j(g)}$. 即内面附近的且相交为余二维面$f$的facets 对应的生成元是彼此相关的.
又因为每对facets pair对应的生成元互为逆元,所以当$F_{i,g_1}\cap F_{i,g_2}= \varnothing$时,$x_{i,g_1}= (x_{i,g_2})^{-1}$. 

最后,我们考虑$F_{i,g}\in \mathcal{F}_3(Q)$的情况. 我们不妨固定$F_{i,1}\in \bar{\pi}^{-1}(F_i)$,对应的生成元为$x_{i,1}$. 首先它的配对facets对应的生成元$x_{i,\phi_i(1)}=(x_{i,1})^{-1}$. 
由于与$F_{i,1}$相交的facets都在$P_{1}$中,所以任意$f=F_{i,1}\cap F_{j,1}\neq \varnothing$对应的关系为$x_{i,1}x_{j,\phi_i(1)}=x_{j,1}x_{i,\phi_j(1)}$,即$x_{i,\phi_j(1)}=x_{i,\lambda(F_j)}=(x_{j,1})^{-1}x_{i,1}x_{j,\phi_i(1)}$.
然后,我们对$F_{i,\phi_j(1)}$进行上面的讨论. 所以$\forall g \in \langle \phi_i(1),\{\phi_j(1)\}\rangle $, $x_{i,g}$都与$x_{i,1}$相关,其中$j\in \{j:F_j\cap F_i \neq \varnothing\}$. 我们仅考虑$F_i$一个顶点处的染色,我们知$\langle \{\phi_j(1)\}\rangle\cong \mathbb{Z}_2^{n}/\langle\phi_i(1)\rangle$. 所以$\langle \phi_i(1),\{\phi_j(1)\}\rangle \cong \mathbb{Z}_2^{n}$,这就证明了所有的$\{x_{i,g}:g\in \mathbb{Z}_2^{n}\}$是相关的.
$\hfill{} \Box$

{\rem 1、注意这里不排除$F_{i,g}\cap F_{i,\phi_j(g)}, F_{i,g}\cap F_{i,\phi_i\phi_j(g)}$都为$Q$中非空的余二维面的情况,此时$(x_{i,\phi_j(g)})^{-1}=x_{i,\phi_i\phi_j(g)}=x_{i,g}=x_{i,\phi_j(g)}$,i.e. $(x_{i,\phi_j(g)})^{2}=1$. 从而$\{x_{i,g}:g\in\mathbb{Z}_2^n\}$为$\pi_1(M)$中的相等的二阶生成元. \\
2、Davis-Januszkiewicz \cite{DJ1} theroem 3.1 中指出small cover Mod 2 Betti 数$b_i(M)=h_i(P)$(这里$h_i$定义中$f_k$表示$P$中余$k+1$维面的个数). $b_1(M)=h_i(P)=m-n$,即在perfect 意义上的胞腔结构得到基本群生成元个数为$m-n$个. 在这里所有外facets 决定的生成元实际上也是$m-n$个. 并且是$\pi_1(M)$最少生成元个数(\ref{lem1}).
%此时$P$可以表示为$\Delta^{n_1}\times Y$,其中$2\leq n_1\leq n$,$ Y$为$n-n_1$维单多面体.
}
%\\{\bf 猜想:}$P$中存在$\Delta^2$面当且仅当$\pi_1(M)$中有二阶元.(必要性易证)from Hampel 3-manifolds page 170 15.1 可以知道对于三维可定向small cover(RP^3除外)的基本群都是torsion free infinite的,非可定向small cover 基本群中有限阶元只有2阶。推测任意维数的small cover 的基本群中元素为有限阶的,则它的阶数只能是2.

在下面例子中,我们只取每个facets pair中的其中一个facets对应的闭路作为基本群的生成元.

%%%%%%%%%%%%%%%%%%%%%%%%%%%%%%%%%%%%%%%%%%%%%%%%%%%%%%
%\section{examples}
{\exmp $P$为五边形时,$\mathcal{F}$上的染色依次取为$\{e_2,e_1e_2,e_1,e_2,e_1\}$,$Q$可视为$12$边形,对应$6$对外facets,$4$组余二维外面。
\begin{figure}[H]
 \centering
 \includegraphics[width=\textwidth]{picture/M5-20.eps}
\put(-170,110){$p_0$}
\put(-150,233){$F_{1,1}$}
\put(-80,220){$F_{2,1}$}
\put(-30,150){$F_{3,1}$}
\put(-215,233){$F_{1,e_1}$}
\put(-290,220){$F_{2,e_1}$}
\put(-340,150){$F_{3,e_1}$}
\put(-150,10){$F_{1,e_2}$}
\put(-80,20){$F_{2,e_2}$}
\put(-20,90){$F_{3,e_2}$}
\put(-215,10){$F_{1,e_1e_2}$}
\put(-290,20){$F_{2,e_1e_2}$}
\put(-350,90){$F_{3,e_1e_2}$}
\end{figure}
%求$M$的基本群。
$Q$中的facets pair 有$\{F_{2,e_1},F_{2,e_2}\}$,$\{F_{1,e_1},F_{1,e_1e_2}\}$,$\{F_{1,1},F_{1,e_2}\}$,$\{F_{2,1},F_{2,e_1e_2}\}$,$\{F_{3,1},F_{3,e_1}\}$,$\{F_{3,e_2},F_{3,e_1e_2}\}$(内部facets pair 对应平凡生成元,我们暂不考虑). 
给所有道路一个指向$p_0$的方向,不妨设$p_0$为基本群基点,取生成元为\\
$\begin{cases}
x_{2,e_1}&\longleftrightarrow \vv{a_{2,e_1}}\cdot(\vv{a_{2,e_2}})^{-1}\\
x_{1,e_1}&\longleftrightarrow \vv{a_{1,e_1}}\cdot(\vv{a_{1,e_1e_2}})^{-1}\\
x_{1,1}&\longleftrightarrow \vv{a_{1,1}}\cdot(\vv{a_{1,e_2}})^{-1}\\
x_{2,1}&\longleftrightarrow \vv{a_{2,1}}\cdot(\vv{a_{2,e_1e_2}})^{-1}\\
x_{3,1}&\longleftrightarrow \vv{a_{3,1}}\cdot(\vv{a_{3,e_1}})^{-1}\\
x_{3,e_2}&\longleftrightarrow \vv{a_{3,e_2}}\cdot(\vv{a_{3,e_1e_2}})^{-1}\\
\end{cases}$
\\
在余$2$维面$p_1,p_2,p_3,p_4$处确定四组关系:\\
在$p_1$处胞腔对应的关系为$x_{1,1}=x_{1,e_1}$;\\
在$p_2$处胞腔对应的关系为$x_{1,1}x_{2,e_2}=x_{2,1}x_{1,e_1e_2}$,即$x_{1,1}(x_{2,e_1})^{-1}=x_{2,1}(x_{1,e_1})^{-1}$;\\
在$p_3$处胞腔对应的关系为$x_{2,1}x_{3,e_1e_2}=x_{3,1}x_{2,e_1}$,即$x_{2,1}(x_{3,e_2})^{-1}=x_{3,1}x_{2,e_1}$;\\
在$p_4$处胞腔对应的关系为$x_{3,1}=x_{3,e_2}$.
%在$p_1$对应的关系为$\vv{a_{1,1}}(\vv{a_{1,e_1}})^{-1}\vv{a_{1,e_1e_2}}(\vv{a_{1,e_2}})^{-1}$,即$x_{1,1}(x_{1,e_1})^{-1}=1$\\
%在$p_2$处胞腔的边界对应$\vv{a_{2,1}}(\vv{a_{1,1}})^{-1}\vv{a_{1,e_2}}(\vv{a_{2,e_2}})^{-1}\vv{a_{2,e_1}}(\vv{a_{1,e_1}})^{-1}\vv{a_{1,e_1e_2}}(\vv{a_{2,e_1e_2}})^{-1}$,即$x_{2,1}(x_{1,1})^{-1}x_{2,e_1}(x_{1,e_1})^{-1}=1$\\
%在$p_3$处胞腔的边界对应$\vv{a_{3,1}}(\vv{a_{2,1}})^{-1}\vv{a_{2,e_1e_2}}(\vv{a_{3,e_1e_2}})^{-1}\vv{a_{3,e_2}}(\vv{a_{2,e_2}})^{-1}\vv{a_{2,e_1}}(\vv{a_{3,e_1}})^{-1}$,即$x_{3,1}(x_{2,1})^{-1}x_{3,e_2}(x_{2,e_1})^{-1}=1$\\
%在$p_4$处胞腔的边界对应$\vv{a_{3,e_2}}(\vv{a_{1,1}})^{-1}\vv{a_{3,e_1}}(\vv{a_{3,e_1e_2}})^{-1}$,即$x_{3,1}(x_{3,e_2})^{-1}=1$

从而
\begin{equation}\label{eq4}
\begin{split}
\pi_1(M)&=\langle x_{2,e_1},x_{1,e_1},x_{1,1},x_{2,1},x_{3,1},x_{3,e_2}|
x_{1,1}(x_{1,e_1})^{-1},x_{3,1}(x_{3,e_2})^{-1},\\&~~~~~~~~~~~~~~~x_{1,1}(x_{2,1})^{-1}x_{1,1}(x_{2,e_1})^{-1}, x_{3,1}(x_{2,1})^{-1}x_{3,1}x_{2,e_1}
\rangle \\
&\cong\langle x_{1,1},x_{2,1},x_{3,1}|x_{1,1}(x_{2,1})^{-1}x_{1,1}x_{3,1}(x_{2,1})^{-1}x_{3,1}
\rangle \\
\end{split}\end{equation}
即
$\pi_1(M)\cong\langle x,y,z|xy^{-1}xzy^{-1}z\rangle$\\}
%%%%%%%%%%%%%%%%%%%%%%%%%%%%%%%%%%%%%%%%%%%%%%%%%%%
%\subsection{Connection with Group of Deck Transformation}
\subsection{Universal Covering Space}
设$\pi:M\longrightarrow P$为单多面体$P$上的 small cover. $P$的facets集为$\mathcal{F}(P)=\{F_1,F_2,\cdots,F_m\}$.
下面我们将构造small cover $\pi:M\longrightarrow P$
的(万有)覆叠空间
\begin{equation}\label{eq5}
\mathcal{M}=Q\times \pi_1(M)/\sim
\end{equation}
$(Q,\nu_1)$的外facet $F_{i,g_1}$与$(Q,\nu_2)$的外facet $F_{j,g_2}$粘当且仅当$i=j,~g_1(g_2)^{-1}=\lambda(F_i),~\nu_1(\nu_2)^{-1}=x_{i,g_1}$(或者等价的$\nu_2(\nu_1)^{-1}=x_{i,g_2}$),其中$\nu_1,~\nu_2\in \pi_1(M)$,Q为上文构造的多面体. 下面为记号方便,我们把$(Q,\nu)$简记为$Q_{\nu}$,$Q_{\nu}$的facet $F_{i,g}$记为$F_{i,g}^{\nu}$. 

下面我们将说明$\mathcal{M}$实际上只与单多面体$P$及$P$的面结构有关. 

%称一个$n$维多面体$P$为right angle orbifold,若它局部等同$\mathbb{R}^n/\mathbb{Z}_2^n$. 
我们首先定义由$P$的面结构决定的{\em right-angle Coxeter group $W_P$}如下:
$$W_P=\langle F_1,\cdots,F_m:F_i^2=1; (F_iF_j)^2=1, \forall F_i,F_j\in \mathcal{F}(P),F_i\cap F_j\neq \varnothing\rangle$$

Davis-Januszkiewicz \cite{DJ1} 中构造了
\begin{equation}\label{eq6}
\mathcal{L}=(P\times W_P)/\sim
\end{equation}
其中$(x_1,g_1)\sim (x_2,g_2)$当且仅当$x_1=x_2$,$g_1(g_2)^{-1}\in \langle F:x\in F,~F\in\mathcal{F}(P)\rangle$. 且由Davis \cite{D1}(Theorem 10.1 and 13.5)知$\mathcal{L}$为单连通的.%see

%我们在Davis,Groups generated by reflections and aspherical manifolds not cover by Euclidean space.知$\mathcal{L}$是单连通的. 我们记$\tau:\mathcal{L}\longrightarrow P$为$P$上的一个covering orbifold,将此时对应的覆叠变换群$D(\mathcal{L},\tau,P)$记为$\hat{\pi}_1(P)$.由于$D(\mathcal{L},\tau,P)$恰好可以表示成由它的面映射生成的Coxeter group $W_P$.
%%%%%%%%%%%%%
%(我们需要证明$\mathcal{L}$是$P$的一个universal covering orbifold.下面在其假设正确下证明,若不正确,,,)
%所以我们有 $\hat{\pi}_1(P)\cong W_P$.
%{\bf 证:}$\tau:\mathcal{L}\longrightarrow P$是$P$的一个universal covering orbifold,所以$\hat{\pi}_1(P)\cong D(\mathcal{L},\tau,P)$,而$D(\mathcal{L},\tau,P)$恰好可以表示成由它的面映射生成Coxeter group $W_P$.

设$\widetilde{\lambda}:\mathbb{Z}_2^m\longrightarrow \mathbb{Z}_2^n$为$\mathcal{F}(P)$上特征映射诱导的群同态,定义映射$\psi$为投射$W\longrightarrow W^{ab}\cong \mathbb{Z}_2^m$
% \longrightarrow \mathbb{Z}_2^n$
和$\widetilde{\lambda}$的复合.
{\lem \label{lem1}设$\pi:M\longrightarrow P$为单多面体$P$上的small cover,则有群短正合列
\begin{equation}\label{exact}
1 \longrightarrow \pi_1(M)\overset{\alpha}{\longrightarrow}W_P\overset{\psi}{\longrightarrow}\mathbb{Z}_2^n \longrightarrow  1
\end{equation}
其中$\pi_1(M)\cong \ker\psi$为$W_P$的子群,$W_P=\pi_1(M)\rtimes \mathbb{Z}_2^n$}\\
{\bf 证:}在Davis-Januszkiewicz \cite{DJ1}中,我们知道有上面正合列成立,且$\pi_1(M)\cong \ker \psi$为$W_P$的正规子群. 不妨设$p_0$附近的facets为$\{F_1,F_2,\cdots,F_n\}$,$\lambda(F_i)=e_i$,考虑$\gamma:\mathbb{Z}_2^n\longrightarrow W_P$,$\gamma(e_i)=F_i, i=1,2,\cdots,n$,则$\psi\circ\gamma=id_{\mathbb{Z}_2^n}$,即上面短正合列是可裂的,故 $W_P=\pi_1(M)\rtimes \mathbb{Z}_2^n$.  $\hfill{} \Box$

%下面我们从fundamental domain 的角度来看,我们下面假设不知道$\pi_1(M)$为small cover $M$的基本群,我们通过证明$\mathcal{M}$是单连通的,即$\Pi:\mathcal{M}\longrightarrow M$为$M$的万有覆叠,得出结论:$\pi_1(M)$的确为small cover $M$的基本群,且可以由$Q$的外facets 对应的面映射生成.
{\lem $\mathcal{L}\cong \mathcal{M}$}\\
%{\bf 证:}由$\mathcal{L}$和$\mathcal{M}$的构造中,它们局部都是通过单多面体$P$的顶点附近的facets做反射得到的,所以我们只需要证明$\mathcal{L}$和$\mathcal{M}$中的$P$存在着某种index 对应即可. $W_P=\pi_1(M)\rtimes \mathbb{Z}_2^n$,所以任意$\omega\in W_P$,存在唯一的$\nu\in\pi_1(M),g\in\mathbb{Z}_2^n$,使得$\omega=\nu g$.
%
% $\mathcal{L}$中$(P,\omega_1)$的面$F_i$与$(P,\omega_2)$的面$F_i$相粘当且仅当$\omega_2(\omega_1)^{-1}=F_i\in W_P$,即$F_i=\nu_1 g_1 (\nu_2 g_2)^{-1}=\nu_1 g_1 (g_2)^{-1}(\nu_2)^{-1}$.
% $\mathcal{M}$中的$(P,g_1)_{\nu_1}$的面$F_{i,g_1}$与$(P,g_2)_{\nu_2}$facet $F_{i,g_2}$粘当且仅当$g_1(g_2)^{-1}=\lambda(F_i),\nu_1(\nu_2)^{-1}=F_{i,g_1} ~or~ \nu_2(\nu_1)^{-1}=F_{i,g_2}$,当且仅当$F_i=\nu_1 g_1 (\nu_2 g_2)^{-1}=\nu_1 g_1 (g_2)^{-1}(\nu_2)^{-1}$
%
%
%为了避免混淆,我们把$W_P$的第$i$个生成元$F_i$记为$\omega_i$.我们下面仅考虑第$1={1'}\cdot {1''}$(分别为$W_P,\pi_1(M),\mathbb{Z}_2^n$中的单位元)个多面体$P$的第$i$个面$F_i$的情况,在$\mathcal{L}$中,它与第$\omega_i$个$P$的面$F_i$粘,不妨设$\omega_i=\nu_1g_1$,其中$\nu_1\in \pi_1(M),g_1\in\mathbb{Z}_2^n$;另一方面在$\mathcal{M}$中,$F_{i,1''}^{1'}$与$F_{i,\lambda(F_i)}^{x_{i,1}}$配对粘在一起. 
%第$1'$个$Q$中的第$1''$个多面体$P$的面$F_i$,与第$x_{i,1}\in\pi_1(M)$个$Q$中的第$\lambda(F_i)$个多面体$P$的面$F_i$粘,
%由于$P$相对于$\mathcal{M}$的覆叠变换群仍为面生成的$W_P$,所以$x_{i,1}\lambda(F_i)=\omega_i$,由于$\omega_i=\nu_1g_1$是唯一的,所以$x_{i,1}=\nu_1,\lambda(F_i)=g_1$,即证. 其他位置的$P$类似,所以$\mathcal{M},\mathcal{L}$局部构造一致,从而为同一个空间。  $\hfill{} \Box$\\
{\bf 证明:}$\mathcal{L}$和$\mathcal{M}$之间的同胚是由分裂短正合列\ref{exact}给出的,
%\begin{equation}\label{exact1} 1 \longrightarrow \pi_1(M)\autorightleftharpoons{\alpha}{\beta} W_P\autorightleftharpoons{\psi}{\gamma}\mathbb{Z}_2^n \longrightarrow  1\end{equation}
\begin{diagram}\label{exact1} 
1&\rTo &\pi_1(M) & \rTo^{\alpha} & W_P &  \pile{\rTo^{\psi} \\ \lTo_{\gamma}}&\mathbb{Z}_2^n &\rTo&1\\
\end{diagram}
其中$\psi\circ\gamma=\text{id}_{\mathbb{Z}_2^n}$. 

我们定义$h:\mathcal{L}\longrightarrow \mathcal{M}$ via. $h(x,\omega)=(x_g,\nu)$,其中$g=\psi(\omega)\in\mathbb{Z}_2^n,~\nu=\beta(\omega)\in \pi_1(M)$是唯一的.

定义$h^{-1}:\mathcal{M}\longrightarrow \mathcal{L}$ via.
$h^{-1}(x_g,\nu)=(x,\gamma(g)\alpha(\nu))$.

$h\circ h^{-1}(x_g,\nu)=h(x,\gamma(g)\alpha(\nu))=(x_g,\nu)$;

$h^{-1}\circ h(x,\omega)=h^{-1}(x_g,\nu)=(x,\gamma(g)\alpha(\nu))=(x,\omega)$.

$h,h^{-1}$局部上为identity,故为局部同胚.
%$h,h^{-1}$的连续性是由于它们作用在$x$的局部为identity.

所以 $\mathcal{L}\cong \mathcal{M}$  $\hfill{} \Box$


%{\bf 注:}一般位置,$\mathcal{L}$中$(P,\omega_1)$的面$F_i$与$(P,\omega_2)$的面$F_i$相粘当且仅当$\omega_2(\omega_1)^{-1}=F_i\in W_P$,即$F_i=\nu_1 g_1 (\nu_2 g_2)^{-1}=\nu_1 g_1 (g_2)^{-1}(\nu_2)^{-1}$.
%$\mathcal{M}$中的$(P,g_1)_{\nu_1}$的面$F_{i,g_1}$与$(P,g_2)_{\nu_2}$facet $F_{i,g_2}$粘当且仅当$g_1(g_2)^{-1}=\lambda(F_i),\nu_1(\nu_2)^{-1}=F_{i,g_1} ~or~ \nu_2(\nu_1)^{-1}=F_{i,g_2}$,当且仅当$F_i=\nu_1 g_1 (\nu_2 g_2)^{-1}=\nu_1 g_1 (g_2)^{-1}(\nu_2)^{-1}$
{\rem 在我们的胞腔构造过程中,设$\{F_1,F_2,\cdots,F_n\}$为顶点$p_0$附近的$n$个facets,$\lambda(F_i)=e_i, i=1,2,\cdots,n$,从而我们可以把$2^n$个$P$的copy在$p_0$处粘在一起得到$Q$,即取$\{F_1,F_2,\cdots,F_n\}$为$Q$的内facets,最后得到基本群的表达形式如(\ref{eq0}). 这在引理\ref{lem1}的短正合列中,等价于$\psi(F_i)=\lambda(F_i)=e_i, i=1,2,\cdots,n$. 
所以
$\psi(F_k)=\lambda(F_k)=\prod  e_i^{\delta_i}=\prod  \lambda(F_i)^{\delta_i}=\prod  \psi( F_i^{\delta_i})$,即$\psi(F_k\left(\prod F_i^{\delta_i}\right)^{-1})=1$,其中$k=1,2,\cdots,m;~i=1,2,\cdots,n$; $\delta_i$表示$\lambda(F_k)$上有无$e_i$分量. $F_1,F_2,\cdots,F_n$在$M$中对应的闭路是可缩的,所以$F_k$对应的闭路是$\ker \psi$的生成元.
我们不妨设$\alpha(x_{k,1})=F_k\left(\prod F_i^{\delta_i}\right)^{-1}$,则$F_k=\alpha(x_{k,1})\gamma(\lambda(F_k))\overset{\Delta}{=}(x_{k,1},~\lambda(F_k))$.
进一步我们设$\varphi:\mathbb{Z}_2^n\longrightarrow Aut(\pi_1(M))$~via ~$\varphi_g(\nu)=\alpha^{-1}(\gamma(g)\alpha(\nu)\gamma(g^{-1}))=\alpha^{-1}(\gamma(g)\alpha(\nu)\gamma(g))$,其中$g\in \mathbb{Z}_2^n, \nu\in \pi_1(M)$. 定义$(\nu_1,g_1)\cdot(\nu_2,g_2)=(\nu_1\varphi_{g_1}(\nu_2),g_1g_2)$. 所以$h$的定义可以从$W_P$的生成元$F_i$延拓到整个群$W_P$中.

%我们可以规定$W_P$的生成元$F_i$可以对应$(x_{i,1},~\lambda(F_i))$,一般地$F_{i}F_{j}=(x_{i,1},\lambda(F_{i}))\cdot(x_{j,1},\lambda(F_{j}))=(x_{i,1}\varphi_{\lambda(F_{i})}(x_{j,1}), \lambda(F_{i})\lambda(F_{j}))$. 特别的若$F_i$为内facet,则$F_{i}F_{j}=(x_{j,e_i},\lambda(F_{i})\lambda(F_{j}))$. 
%其中这里的乘积运算可以看为$\pi_1(M),\mathbb{Z}_2^n$看作$\pi_*,\widetilde{\psi}$(满足短正合列可裂的$\mathbb{Z}_2^n\longrightarrow W_P$的映射)作用下在$W_P$中的群运算.
}




接下来我们将证明$\mathcal{M}$为$M$的万有覆叠空间.
$\mathcal{M}$到$M$有一个自然的投射,我们记为$\Pi:\mathcal{M}\longrightarrow M$.
\begin{diagram}
Q\times \pi_1(M) &\rTo^{q'} &Q\times \pi_1(M)/\sim=\mathcal{M}\\
\dTo^{\widetilde{\Pi}}&     &   \dTo^{\Pi}\\
Q   &\rTo^{q}&Q/\sim=M
\end{diagram}
其中$q, q'$是粘合$Q$和$Q$的copy的facets决定的商映射. 


%下面我们说明,small cover 中的任意一个点在某种意义上是地位是一样的.
%{\lem 设$\pi:M\longrightarrow P$为一个固定的small cover,则$\forall x\in M$,$M$可以在点$x$处分解成$2^n$个同构于$P$的多面体.}\\
%{\bf 证明:}%我们同样先把$2^n$个多面体$P$的copy在它们的一个顶点$p_0$出粘在一起,得到一个大的多面体$Q$
%我们考虑商映射$q:Q\longrightarrow M$,这里不妨设$Q$是凸的.  
%若$q^{-1}(x)\subset Q^{\circ}$,则$q^{-1}(x)$为单元集,不妨设$y=q^{-1}(x)$.
%我们在$Q$的内部将点$p_0$连同它附近的$Q$的内面线性地拉到到点$x$处,则此时$Q$可以看为点$x$附近的$2^n$个$P$的copy粘成的. 

%若$q^{-1}(x)\subset \partial Q$,任取$y\in q^{-1}(x)$,我们记$f(y)$为$out(Q)$中包含$y$为相对内点的最小的面,不妨设$f(y)$为余$k$维的,则所有的$\{f(y):y\in q^{-1}(x)\}$都是identity,且$|(q')^{-1}(x)|=2^k$. 事实上,商映射$q$对$\partial Q$上点的局部作用就是将$q^{-1}(x)$中的点连同包含这些点为相对内点的最小的面粘在一起.
%接下来我们将$Q$重新分解成$P$的copy,并将它们在$f(y)$的某个顶点处粘在一起,得到一个大的多面体,记为$\mathop{{Q}'}$,将$Q'$的外facets按照染色信息成对粘在一起,得到同样的small cover $M$. 此时$x$在$Q'$中的原象位于$Q'$的内部,进行上面讨论.  $\hfill{} \Box$

{\thm $\mathcal{M}$为$M$的万有覆叠空间.}\\
%{\bf 证明:}根据上面引理,我们不妨考虑点$x=\pi^{-1}(p_0)\in M$,则$q^{-1}(x)\subset Q^{\circ}$为单元集,我们可以取包含$q^{-1}(x)$的$n$维实心开球$U$,满足$U\subset Q^{\circ}$. 
%则$q(U)$为$M$中包含$x$的开邻域,与$U$为identity. 且$(\widetilde{\Pi})^{-1}(U)$为$|\pi_1(M)|$个互不相交开球的并,即
%$$(\widetilde{\Pi})^{-1}(U)=\bigsqcup\limits_{\nu\in \pi_1(M)}V_\nu$$
%其中每个$V_\nu\subset (Q_\nu)^{\circ}$与$U$为identity. 则
%$$(\Pi)^{-1}(q(U))=q'((\widetilde{\Pi})^{-1}(U))=\bigsqcup\limits_{\nu\in \pi_1(M)}q'(V_\nu)$$
%为一族互不相交的开集,且$\Pi$限制在每一个$q'(V_\nu)$上都为到$q(U)$的identity. 
%
%当$q^{-1}(x)\subset \partial Q$时,我们将$Q$换成$Q'$,得到的$\mathcal{M}$实际上是不变的,这是因为$\mathcal{M}$是由多面体$P$决定的. 所以我们可以类似进行上面的操作.
%
%故$\mathcal{M}$为$M$的覆叠空间. 又因为$\mathcal{M}\cong\mathcal{L}$为单连通的,故为万有覆叠空间.  $\hfill{} \Box$
{\bf 证明:}由$\mathcal{M}$的定义知$\mathcal{M}/\pi_1(M)=M$.
下面只需要证明$\forall x\in \mathcal{M}$,存在包含点$x$的开邻域$U$,使得不同的$\nu\in\pi_1(M)$,$\nu(U)$不交.

当点$x\in Q^\circ$时,取包含点$x$的一个实心开球邻域$U$,使得$U\subset Q^\circ$,此时$\nu(U)$落在不同指标的$Q$中,是不交的.
当点$x\in out(Q)=\partial Q$时,我们取

若$q^{-1}(x)\subset \partial Q$,我们记$f(x)$为$out(Q)$中包含$x$为相对内点的最小的面,不妨设$f(x)$为余$k$维的,则我们取包含点$x$的实心开球邻域$U$,满足$U$与包含点$x$的$Q$的copy的交都是$\frac{1}{2^k}$球,这时$\nu(U)$显然也是不交的.

所以$\mathcal{M}$是$M$一个正则的覆叠空间,又$\mathcal{L}\cong \mathcal{M}$是单连通的,故为万有覆叠空间.
$\hfill{} \Box$


设$D(\mathcal{M},\Pi,M)$为上面覆叠空间$\Pi: \mathcal{M}\longrightarrow M$的覆叠变换群. 由于$\mathcal{M}$是单连通的,所以$\pi_1(M)\cong D(\mathcal{M},\Pi,M)$. 
下面我们根据上面构造的cell structure(的$2$-skeleton)来刻画$D(\mathcal{M},\Pi,M)$的生成元. 
%这本质上是利用基本域的想法描述离散群.

对于$Q$中的每个facet $F_{i,g}$,我们定义$\mathcal{M}$上的{\em 面映射}$\Gamma_{i,g}:\mathcal{M}\longrightarrow \mathcal{M}$. $\forall x\in \mathcal{M}$,存在某个$Q_{\nu_1}$,使得$x\in Q_{\nu_1}$,由$\mathcal{M}$的构造知存在唯一的$Q_{\nu_2}$,使得$F_{i,g}\subset Q_{\nu_1}\cap Q_{\nu_2}$,我们定义$\Gamma_{i,g}(x)$为$\Pi^{-1}(\Pi(x))\cap Q_{\nu_2}$中的唯一的一点,这样定义的$\Gamma_{j,g'}$显然是well-defined的. $\Gamma_{i,g}$的连续性也是显然的.

类似引理\ref{lem3}容易验证

{\lem 1、$\Gamma_{i,g}\Gamma_{i,\phi_i(g)}(x)=x$. \\
2、若存在$F_{j,g'}(\neq F_{i,g})\subset Q_{\nu_1}\cap Q_{\nu_2}$,则$\Gamma_{j,g'}(x)=\Gamma_{i,g}(x)$. \\
3、若facet $F_{i,g}\in in(Q)$,此时$Q_{\nu_1}=Q_{\nu_2},\Gamma_{i,g}=id.$}

进一步我们有
{\lem 面映射$\Gamma_{i,g}:\mathcal{M}\longrightarrow \mathcal{M}$为$\mathcal{M}$上的覆叠变换.
}



{\prop $D(\mathcal{M},\Pi,M)$可以由面映射$\{\Gamma_{i,g}\}$来刻画.}\\
{\bf 证明:}$\mathcal{M}$为单连通的,此时$\pi_1(M,p_0)$到$D(\mathcal{M},\Pi,M)$的满同态实际上为群同构,它将$[x_{i,g}]\in \pi_1(M,p_0)$映为$\Gamma_{i,g}$. 
我们不妨取点$x=\pi^{-1}(p_0)\in M$,则$\{\Pi^{-1}(x)\}$实际上是每个$Q$中$p_0$的copy. 我们取第$1$个$Q_1$中的$p_0$的copy,记为$y_0$,其中$1$为$\pi_1(M)$的单位元. 所以我们只需要验证$\Gamma_{i,g}(y_0)=\widetilde{x_{i,g}}(1)$,其中$\widetilde{x_{i,g}}$是$x_{i,g}$在$\mathcal{M}$中的一段提升.
%,则$\{\Pi^{-1}(x)\}=\{y=h(y_0):h\in \pi_1(M)\}$
$\Gamma_{i,g}(y_0)$实际上是$Q_{x_{i,g}}$中的$p_0$的copy,即$\widetilde{x_{i,g}}(1)$. 所以$D(\mathcal{M},\Pi,M)$的生成元可以自然的选为$Q$的facets 对应的面映射.  $\hfill{} \Box$
%另外容易看到$\Pi^{-1}(x_{i,g})$在$\mathcal{M}$中的一段提升是连接$y_0$与$\Gamma_{i,g}(y_0)$的一条道路设为$\gamma$,即$\Gamma_{i,g}(y_0)=\gamma_{\#} y_0$,则$\Pi_*(\pi_1(\mathcal{M},\Gamma_{i,g}(y_0)))=\Pi_*(\pi_1(\mathcal{M},\gamma_{\#}y_0))=\Pi_*((\gamma)^{-1}\pi_1(\mathcal{M}, y_0)\gamma)=\Pi_*((\gamma)^{-1})\Pi_*(\pi_1(\mathcal{M}, y_0))\Pi_*(\gamma)=x_{i,g}^{-1}\Pi_*(\pi_1(\mathcal{M}, y_0))x_{i,g}=x_{i,\phi_i(g)}\Pi_*(\pi_1(\mathcal{M}, y_0))x_{i,g}$

%任取一点$y\in (q')^{-1}(x)$,取$U_y=Q\cap D(y,\epsilon)$,其中$D(y,\epsilon)$以$y$为圆心,足够小的$\epsilon$为半径的$n$维实心开球. 则$q(U)$为$M$中包含点$x$的$n$维实心开球,其中$U=\bigsqcup\limits_{y\in (q')^{-1}(x)}U_y$.事实上记$f(y)$为$out(Q)$中包含$y$为相对内点的最小的面,不妨设为余$k$维的,则$|(q')^{-1}(x)|=2^k$,每个$U_y$实际上是小开球的$\frac{1}{2^k}$,它们在$q$的作用下粘成一个整开球.进一步,$$(\widetilde{\Pi})^{-1}(U)=\bigsqcup\limits_{\nu\in \pi_1(M)}V_\nu$$且每个$V_\nu\subset (Q_\nu)$与$U$为identity.由$\mathcal{M}$的构造知,$q'(\widetilde{\Pi})^{-1}(U)$实际上是将不同$Q_\nu$的$V_\nu$也粘为$|\pi_1(M)|$个互不相交的$n$维实心开球,且每个开球与$q(U)$是identity的.这就证明了$\Pi:\mathcal{M}\longrightarrow M$为一个覆叠空间. 

%下面说明它是正则的.不妨取$x_0=q(p_0)\in M$,任意$y\in (\Pi)^{-1}(x_0)$

%取$p'_0\in q'\subset \mathcal{M}$,其中这里的$1$为$\pi_1(M)$的单位元,满足$\Pi(p'_0)=p_0$,则$\Pi^{-1}(p_0)$与$\{h(p'_0):h\in \pi_1(M)\}$是一一对应的,进一步若$\mathcal{M}$单连通,则$D(\mathcal{M},\Pi,M)\cong \pi_1(M)$,且$D(\mathcal{M},\Pi,M)$就是由上面facets 对应的面映射生成的,即我们构造的cell structure 自然对应于它的万有覆叠的覆叠变换群.






综上,我们有下面结论:
{\thm $\Pi:\mathcal{M}\longrightarrow M$为$M$的万有覆叠空间,复叠变换群$D(\mathcal{L},\Pi,M)\cong\pi_1(M)$可以由$Q$的facets对应的面映射生成.}


\subsection{Naturality of the Cell Structure of Small Cover}
最后我们解释一下我们这种胞腔结构的自然. 我们考虑$M[2]$,它的$0$-skeleton只有一个点$p_0$; 它的$1$-skeleton 是$\vv{x_{i,g}}$的一点并,对应$\pi_1(M)$的生成元; 它的每一个二维胞腔对应$\pi_1(M)$的一个关系. 即$M[2]$是$\pi_1(M,p_0)$的{\em presentation complex}.
进一步,我们将$M$的这种胞腔结构提升到它的万有覆叠空间$\mathcal{M}$中,则$\mathcal{M}[2]$实际上是$\pi_1(M,p_0)$的{\em Cayley $2$-complex.}

事实上,将单多面体$P$视为一个 right angle orbifold,则small cover $\pi:M\longrightarrow P$为$P$上的covering orbifold. 
由于$M$为一个闭流形,$P$为一个good orbifod,则$P$的单连通的covering orbifold $\mathcal{M}$为它的万有covering orbifold.
%我们可以将$Q$看为关于群$\pi_1(M)$的基本域,$P$看为关于coxeter群$W_P$的基本域. 即
进一步,
covering orbifold $\tau:\mathcal{M}\longrightarrow P$ 为covering space $\Pi:\mathcal{M}\longrightarrow M$ 和small cover $\pi:M\longrightarrow P$的复合. 
它们的覆叠变换群分别为$W_P, \pi_1(M)$ 和$\mathbb{Z}_2^n$.(Davis-Januszkiewicz \cite{DJ1})

我们定义$\pi_1^{orb}$为universal orbifold cover $\tau:\mathcal{M}\longrightarrow P$的覆叠变换群.
即$\pi_1^{orb}(P)=W_P$,
此时$\pi_1^{orb}(P)$和$\pi_1(M)$存在自然的子群关系.
我们对单多面体$P$做类似的cubical 分解,则某种意义上,$P[2]$为$W_P$的presentation complex. 把这种分解提升到$\mathcal{M}$中,则$\mathcal[2]$实际上是$W_P$的一个 Cayley 2-complex. (Davis \cite{D2})

%我们取$\mathcal{F}_2(Q)$中的任意一个facet $F$,诱导small cover 记为$M_F$,设与$F$横截相交的一维face为$F'$,诱导的small cover ($S^1$) 记为$M_{F'}$. 则$M_F$与$M_{F'}$是横截相交的,所以$[M_F]$的对偶上同调类与$[M_{F'}]$的对偶上同调类的cup 积是$[M_{F\cap F'}]$的对偶. 进一步对任意余$k$维面$f=F_1\cap F_2 \cap \cdots \cap F_k$,$M_f$对应的上同调类,都可以由一维上同调类生成. 即这里我们可以取上同调环$H^*(M)$的生成元为外facets对应的闭路对应的$1$维上同调类.(Davis-Januszkiewicz \cite{DJ1})
 


%%%%%%%%%%%%%%%%%%%%%%%%%%%%%%%%%%%%%%
\section{$\pi_1$-injective}
设$F$为单多面体$P$的任意一个facet,则它依然是单凸的,且$\mathcal{F}(F)$可以继承$\mathcal{F}(P)$上的染色,进而可以构造$F$上的small cover $\pi_F:M_F\longrightarrow F$. 在Davis-Januszkiewicz \cite{DJ1} Lemma 1.3 中,我们知道$M_F$为$M$的$n-1$维连通子流形. 

在这一节中,我们利用上面的胞腔结构,考虑$\pi_1(M_F)$与$\pi_1(M)$之间的关系. 我们设多面体$F,P$对应的Coxeter group 分别为$W_P=\langle G:R \rangle,~W_F=\langle G_F:R_F\rangle$.
我们取定$F$的一个顶点$p_0$为基点,我们将多面体$P$的copy在$p_0$处粘在一起,从我们构造的胞腔结构中,得到$M,M_F$的基本群的表示,如(\ref{eq0}),记$\pi_1(M,p_0)=\langle \widetilde{G}:\widetilde{R}\rangle~\pi_1(M_F,p_0)=\langle \widetilde{G}_F:\widetilde{R}_F\rangle$. 

下面我们设$\rho:F\longrightarrow P$为面包含映射,$\rho_*:W_F\longrightarrow W$为$\rho$决定的Coxeter group之间的群同态,设$\widetilde{\rho}:M_F\longrightarrow M$为$\rho$决定的Facet子流形到$M$的包含映射,$\widetilde{\rho}_*:\pi_1(M_F)\longrightarrow\pi_1(M)$为它诱导的基本群之间的群同态.
%设$F^k$为单凸多面体$P$的一个$k$维面,$F^k$仍为单凸的,且继承了$P$染色.记$F^{k}$上的 small cover 为$M_F$.
%下面在上面 cell structure 下,分析$\pi_1(M_F)$与$\pi_1(M)$的关系.
下面不妨设$F$为单多面体$P$的第$1$个facet,$\lambda(F)=e_1$.
%并取定$F$的一个顶点$p_0$.
%我们记$W_F,~W_P$分别为$F$和$P$的Coxeter group,$\pi_F:M_F\longrightarrow F$和$\pi_P:M\longrightarrow P$分别为$F$和$P$上的small cover. 
%设$\rho:F\longrightarrow P$为面包含映射,$\hat{\rho}:W_F \longrightarrow W_P$为$\rho$诱导的Coxeter group之间的群同态,$\rho_{*}:\pi_1(M_F,p_0) \longrightarrow \pi_1(M,p_0)$为$\rho$诱导的small cover基本群之间的群同态.
%另外我们把$W_P$的生成元集$\mathcal{F}(P)$记为$\widetilde{G}_P$,关系集记为$\widetilde{R}_P$;
%把$W_F$的生成元集$\mathcal{F}(F)$记为$\widetilde{G}_F$,关系集记为$\widetilde{R}_F$.
%取$p_0$为基点,按照引理中的方式,分别得到$\pi_1(W_F,p_0)$和$\pi_1(W_P,p_0)$的群表示,分别记它们生成元集为$G_F, G_P$,关系集为$R_F,R_P$. 设 

%{\lem (1)、\label{lemm} $G_F\subset G, R_F\subset R$,且$\rho_{*}(G_F)=id$; \\
%(2)、$\widetilde{G}_F\subset \widetilde{G}$,$\widetilde{R}_F\subset \widetilde{R}$,且$\widetilde{\rho}_*|_{\widetilde{G}_F}=id$. 
%}\\
{\lem \label{lemm}1、 $G_F$与$\rho_{*}(G_F)$,$R_F$与$\rho_{*}(R_F)$都为一一的;\\
2、
$\widetilde{G}_F$与$\widetilde{\rho}_*(\widetilde{G}_F)$,$\widetilde{R}_F$与$\widetilde{\rho}_*(\widetilde{R}_F)$都为一一的.
}\\
{\bf 证明:}
%%%%%%%%%Coxeter group 部分 最好添加个图
1、设$P$中与$F$相交的Facets集为$\mathcal{F}_1=\{F_{i_1},F_{i_2},\cdots,F_{i_{m'}}\}\subset \mathcal{F}(P)$,则$G_F=\mathcal{F}(F)=\{F_{i_1}\cap F,F_{i_2}\cap F,\cdots,F_{i_{m'}}\cap F\}$. 所以自然地,$\rho_{*}(F_{i_k}\cap F)=F_{i_k}$. 进一步$F$中余二维面$(F_{i_{k_1}}\cap F)\cap (F_{i_{k_2}}\cap F)\neq\varnothing$决定的关系在$\rho_{*}$在的像自然对应$F_{i_{k_1}}\cap F_{i_{k_2}}$在$W$中决定的关系. 
即$\rho_{*}$把生成元映成生成元,把关系映成关系.

\begin{figure}[h]
\centering
\def\svgwidth{0.5\textwidth}
\input{FF.pdf_tex}
\end{figure}
%%%%%%%%%Fundamental Group 部分
2、我们分别将$\{(F,g)\}_{g\in \mathbb{Z}_2^{n-1}}$与$\{(P,g)\}_{g\in \mathbb{Z}_2^{n}}$在点$p_0$处粘合在一起,分别得到多面体$Q$与$Q_F=F\times \mathbb{Z}_2^{n-1}/\sim$.则$out(Q_F)\subset out(Q),in(Q_F)\subset in(Q)$. 设$f_{i}=F_i\cap F\neq \varnothing$为$F$的一个任意的facet,$f_i\cap f_j=F_i\cap F_j\cap F\neq \varnothing$为$F$的一个任意的余2维面. 设$f_{i,g}=F_{i,g}\cap F_{1,g}=F_{i,\phi_i(g)}\cap F_{1,\phi_i(g)}$为$Q_F$中的任意一个facet,其中$\{F_{1,g},F_{1,\phi_i(g)}\}$为$\{P_g: g\in \mathbb{Z}_2^n\}$中的facets-pair. 由引理\ref{lem3}~,我们知道$f_{i,g}$在$Q_F$中对应的有向闭路与$F_{i,g}$和$F_{i,\phi_i(g)}$在$Q$中对应的有向闭路$x_{i,g},x_{i,\phi_i(g)}$是定点同伦的,所以我们不妨记$f_{i,g}$在$Q_F$中对应的有向闭路为$x_{i,g}$.对于$Q_F$中的任意一个余$2$维面$f_{i,g}\cap f_{j,g}=F_{i,g}\cap F_{j,g}\cap F_{1,g}\neq\varnothing$所对应的二维胞腔$D_l$与$F_{i,g}\cap F_{j,g}$和$F_{i,\phi_i(g)}\cap F_{j,\phi_i(g)}$所对应的二维胞腔$D_{g},D_{\phi_i(g)}$是也是定点同伦的,所以在$\pi_1(M_F)$中,$f_{i,g}\cap f_{j,g}$决定的关系与$F_{i,g}\cap F_{j,g}(\cap F_{1,g}\neq \varnothing)$或者$F_{i,\phi_i(g)}\cap F_{j,\phi_i(g)}(\cap F_{1,\phi_i(g)}\neq \varnothing)$在$\pi_1(M)$中决定的关系对应。
所以$M_F$的基本群为
\begin{multline}
\pi_1(M_F)=\langle x_{i,g},i=1,2,\cdots,m',g\in \mathbb{Z}_2^{n-1}:x_{i,g}x_{i,\phi_i(g)}=1,\forall i,g\\
x_{i,g}x_{j,\phi_i(g)}x_{i,\phi_i\phi_j(g)}x_{j,\phi_j(g)}=1,\forall f_{i,g}\cap f_{j,g}\neq \varnothing\rangle
\end{multline}
其中$f_{i,g}\cap f_{j,g}=F_{i,g}\cap F_{j,g}\cap F_{1,g}=F_{i,\phi_i(g)}\cap F_{j,\phi_i(g)}\cap F_{1,\phi_i(g)}\neq \varnothing$.
即形式上$\pi_1(M_F)$的生成元集$\widetilde{G}_F$和关系集$\widetilde{R}_F$都可为$\pi_1(M)$的生成元集$\widetilde{G}$和关系集$\widetilde{R}$的子集. 进一步,这种关系是映射$\widetilde{\rho}:M_F\longrightarrow M$所诱导的,即
$\widetilde{\rho}_{*}$把生成元映为生成元,把关系映为关系.$\hfill{} \Box$

%有$\widetilde{\rho}_{*}|_{\widetilde{G}_F}=id$.  $\hfill{} \Box$

进一步对一般的$k$维面$f$,由归纳知,则$W_f$和$W$,$\pi_1(M_f)$和$\pi_1(M)$都有上面的关系. 不妨仍记$\rho:f\longrightarrow P$为面包含映射,$\rho_*:W_f\longrightarrow W$为Coxeter group之间的群同态,$\widetilde{\rho}_*:\pi_1(M_f,p_0)\longrightarrow \pi_1(M,p_0)$为基本群之间的群同态,则
{\cor 1、 $G_f$与$\rho_{*}(G_f)$,$R_f$与$\rho_{*}(R_f)$都为一一的;\\
2、
$\widetilde{G}_f$与$\widetilde{\rho}_*(\widetilde{G}_f)$,$\widetilde{R}_f$与$\widetilde{\rho}_*(\widetilde{R}_f)$都为一一的.
}

事实上$\rho_*(F_i\cap F)=F_iF^{-1}$,我们人为得在$G_F,R_F$中加入$F$,则$W_F=\langle G_F,F:R_F,F\rangle\cong \langle G_FF^{-1},F:R'_F,F\rangle=\langle G_FF^{-1}:R'_F\rangle\overset{\Delta}{=}W'_F$,则$\rho_*$对$W'_F$中的生成元和关系的作用就是包含到$G,R$中.在基本群中内facets对应的生成元实际上就是这种作用.
所以我们不妨就记$W_F$的生成元为$\rho_*(G)$,$\pi_1(M_F)$的生成元为$\widetilde{\rho}_*(\widetilde{G})$. 
即$W_F$可以用$W$中的生成元和关系来表示,$\pi_1(M_F)$可以用$\pi_1(M)$中的生成元和关系来表示.但一般$\rho_{*}$与$\widetilde{\rho}_*$不一定是单同态. 如 
{\exmp 取$P=I\times \triangle ^2$为三棱柱,共有$5$个facets $\{F_i\}_{i=1,2,3,4,5}$,我们给上下底面$F_1,F_2$染色$e_1$,侧面$F_3,F_4,F_5$染色为$e_2,e_3,e_1e_2e_3$,由$P$的$h$-vector 知,$\pi_1(M)$有两个生成元和两个关系,它的任意一个侧面上的small cover基本群有两个生成元,一个关系.

即$\pi_1(M)=\langle x,y:x^2=yxyx^{-1}=1\rangle$,$\pi_1(M_F)=\langle x,y:yxyx^{-1}=1\rangle$\\
\begin{diagram}
\widetilde{\rho}_*:&\pi_1(M_F)&\rTo &\pi_1(M)\\
\end{diagram}
满足$\widetilde{\rho}_*(x)=x,\widetilde{\rho}_*(y)=y$,但$\widetilde{\rho}_*$非单.

此时$\rho_*$也不是单的,不妨考虑$F_5$,则$F_3\cap F_4$在$W$中决定的关系$(F_3F_4)^2=1$是$((F_3\cap F_5)(F_4\cap F_5))^2(\neq 1)\in W_{F_5}$在$\rho_*$下的像. 
}

%下面我们简要介绍small cover 中的Borel conjecture.
{\defn
我们称一个单纯复形$K$为{\em flag}的,如果$K$中两两相连的顶点集张成$K$中的一个单形. 等价地,$K$中不含维数$\geq 2$的空单形.

我们称一个单多面体$P$为flag的,如果$K=\partial P$为flag的. 等价地,$P$中两两相交的面必有公共的交.}

{\exmp 1、一个$m$边形为flag的,当且仅当$m>3$.

2、flag 多面体的面是flag 的.
}

%(下面证明有问题!!!)
%命题
%{\thm 当多面体$P$为flag时,$\rho_{*}$为单同态.}\\
%证明:多面体$P$为flag的,是指$P$中两两相交的facets必有公共的交.
%{\bf 证明:}当多面体$P$为flag时,则在引理中若$F\cap F_i\neq \varnothing, F\cap F_j\neq \varnothing$,则$F\cap F_i\cap F_j\neq \varnothing$,即任意$F$附近的余二维面$f\subset Q$对应的关系一定可以继承到$M_F$的基本群中.这保证了下面定义的态射的合理性.
%我们构造态射
%\begin{equation}
%\eta_{*}:\pi_1(M)\longrightarrow \pi_1(M_F)
%\end{equation}
%满足$\eta_{*}|_{G-G_F}=1$,$\eta_{*}|_{G_F}=id$. 

%下面我们验证$\eta_{*}$定义的合理性. 考虑$\pi_1(M)$中关系在$\eta_{*}$下的像. $Q$中与$(\bar{\pi})^{-1}(F)$相交的facets集(包含$(\bar{\pi})^{-1}(F)$),我们记为$\mathcal{F}_1$,与$\mathcal{F}_1$中facets相交且不包含$(\bar{\pi})^{-1}(F)$的facets集,我们记为$\mathcal{F}_2$,剩余的facets我们记为$\mathcal{F}_3$.  
%则$\eta_{*}$将$\mathcal{F}_2, \mathcal{F}_3$中facets对应的生成元映为$1$.
%对于$Q$中的任意余二位面$f=F_{i,g}\cap F_{j,g}$,若$F_{i,g},F_{j,g}$都属于$\mathcal{F}_1$,则由$P$的flag性质知$f\cap F\neq \varnothing$,
%从而$\eta_{*}$将$f\subset Q$所对应的关系映为$\pi_1(M_F)$的一个关系;
%则对应的关系在$\eta_{*}$下的像不变,为$\pi_1(M_F)$中的关系;
%若$F_{i,g},F_{j,g}$都不属于$\mathcal{F}_1(F)$,则对应关系在$\eta_{*}$下的像为$1$;
%若$F_{i,g},F_{j,g}$分别属于$\mathcal{F}_1(F),\mathcal{F}_2(F)$,
%不妨设$F_{i,g}\subset \mathcal{F}_1(F),F_{j,g}\subset \mathcal{F}_2(F)$,
%设这个关系为$x_{i,g}x_{j,\phi_i(g)}x_{i,\phi_i\phi_j(g)}x_{j,\phi_j(g)}=x_{i,g}x_{i,\phi_i\phi_j(g)}=1$,
%则$\eta_{*}(x_{i,g}x_{j,\phi_i(g)}x_{i,\phi_i\phi_j(g)}x_{j,\phi_j(g)})=\eta_{*}(x_{i,g})\eta_{*}(x_{i,\phi_i\phi_j(g)})=x_{i,g}x_{i,\phi_i\phi_j(g)}=1$. 对于$Q$中的facets pair,它们同时属于或不属于$\mathcal{F}_1(F)$,
%所以$\pi_1(M)$中的配对关系,在$\eta_{*}$下的像要么不变要么为$1$. 
%所以对任意关系$r\in \pi_1(M),\eta_{*}(r)\equiv 1$,即$\eta_{*}$ 为well-defined.

%进一步,对任意word $w=\Game(x:x\in G_F)\in \pi_1(M_F)$,$\eta_{*}\rho_{*}(w)=\eta_{*}\rho_{*}(\Game(x:x\in G_F))=\eta_{*}(\Game(x:x\in G))=\Game(x:x\in G_F)=w\in \pi_1(M_F)$,即$\eta_{*}\rho_{*}=id:\pi_1(M_F)\longrightarrow \pi_1(M_F)$,故$\rho_{*}$单.   $\hfill{} \Box$


%或对任意word$l=\Game(x:x\in G_F)=1\in \pi_1(M)$,则$l=\Im(r:r\in G)$,$\Game(x:x\in G_F)=\eta_{*}(\Game(x:x\in G_F))=\eta_{*}(g)=\eta_{*}(\Im(r:r\in G))=\Im(\eta_{*}(r):r\in G)=\Im(r:r\in G_F)=1$,则知$\rho_{*}$为单.

%事实上,我们令$\mathcal{F}^2(F)\cup \mathcal{F}^3(F)$中的facets 染色都为$\mathbb{Z}_2^n$的单位元$1$,则得到空间$\widetilde{M_F}=P\times \mathbb{Z}_2^n/\dot{\sim}$,与$M_F$是同伦等价的. 




下面证明当$P$为flag时,$\rho_{*}$和$\widetilde{\rho}_*$都是单的.

%{\lem $\widetilde{G}_F\subset \widetilde{G}_P$,$\widetilde{R}_F\subset \widetilde{R}_P$,进一步$\hat{\rho}|_{\widetilde{G}_F}=id$.}\\
%{\bf 证明:}类似上面引理的证明.

{\lem 当$P$为flag时,$\rho_{*}:W_F\longrightarrow W_P$为单的.}\\
{\bf 证明:}
%若$F\cap F_i\neq \varnothing, F\cap F_j\neq \varnothing$,则$F\cap F_i\cap F_j\neq \varnothing$,即任意$F$附近的余二维面$f\subset P$对应的关系一定可以继承到$W_P$中.
当单多面体$P$为flag时,若$F\cap F_i\neq \varnothing, F\cap F_j\neq \varnothing$,则$F\cap F_i\cap F_j\neq \varnothing$,即$P$中$F$附近的任意余二维面$f\subset P$对应的关系一定可以继承到$W_F$中.这保证了下面这个态射的定义合理性.
我们构造群态射
\begin{equation}\label{eq7}
\eta_{*}:W_P\longrightarrow W_F
\end{equation}
满足$\eta_{*}|_{G-G_F}=1$,$\eta_{*}|_{G_F}=id$. 

下面我们验证$\eta_{*}$定义的合理性. 我们考虑$W_P$中的关系在$\eta_{*}$下的像是否为$W_F$的单位元. 
设$P$中与$F$相交的facets集(不包含$F$)为$\mathcal{F}_1(=G_F)$,与$\mathcal{F}_1$中facets相交且不包含$F$的facets集,我们记为$\mathcal{F}_2$,剩余的facets我们记为$\mathcal{F}_3$.  
则$\eta_{*}$将$\mathcal{F}_2, \mathcal{F}_3$中facets对应的生成元映为$(F_iF_j)^2=1$.

对于$W_P$的关系$(F_i)^2=1$,当$F_i\in \mathcal{F}_1$时,$\eta_{*}(F_iF_i)=(F_i)^2=1$;当$F_i\in \mathcal{F}(F)-\mathcal{F}_1$时,$\eta_{*}(F_iF_i)=1$. 

$P$中的任意余二维面$f=F_{i}\cap F_{j}\neq\varnothing$决定的关系为$(F_iF_j)^2=1$.若$F_{i},F_{j}$都属于$\mathcal{F}_1$,则由$P$的flag性质知$f\subset F$,
从而$\eta_{*}$将$f\subset P$所对应的关系映为$W_F$的一个关系;
若$F_{i},F_{j}$都不属于$\mathcal{F}_1$,则对应关系在$\eta_{*}$下的像为$1$;
若$F_{i},F_{j}$分别属于$\mathcal{F}_1,\mathcal{F}_2$,
不妨设$F_{i}\subset \mathcal{F}_1,F_{j}\subset \mathcal{F}_2$,
则$\eta_{*}(F_iF_jF_iF_j)=\eta_{*}(F_i)\eta_{*}(F_j)\eta_{*}(F_i)\eta_{*}(F_j)=(F_i)^2=1$. 

所以对任意关系$r\in W_P,\eta_{*}(r)\equiv 1$,即$\eta_{*}$ 为well-defined的群同态.

最后容易验证
$\eta_{*}\circ\rho_{*}=id:W_F\longrightarrow W_F$,即$\rho_{*}$为单的.  $\hfill{} \Box$


{\thm $P$为flag时,$\widetilde{\rho}_*:\pi_1(M_F,p_0)\longrightarrow \pi_1(M,p_0)$为单同态.}\\
{\bf 证明:}
考虑pull back 
\begin{diagram}
M_F &\rTo^{\tilde{\rho}} &M\\
\dTo^{\pi_F}&     &   \dTo^{\pi}\\
F   &\rTo^{\rho}&P
\end{diagram}
则考虑
\begin{diagram}
\pi_1(M_F) &\rTo^{\widetilde{\rho}_*} &\pi_1(M)\\
\dTo^{(\pi_F)_{*}}&     &   \dTo^{\pi_{*}}\\
W_F   &\rTo^{\rho_{*}}&W_P
\end{diagram}
 设$x_k$为$\pi_1(M_F)$的$F_k\cap F$对应的一个生成元,则$\pi_{*}\circ\widetilde{\rho}_*(x_k)=\pi_*(x_k)=F_k(\prod F_i^{\delta_i})^{-1}$;
$\rho_{*}\circ(\pi_F)_{*}(x_k)=\rho_{*}(F_k(\prod F_i^{\delta_i})^{-1})=F_k(\prod F_i^{\delta_i})^{-1}$.
所以上面图表是可交换的. 

所以当$P$为flag时,$\rho_{*}$为单的,所以$\rho_{*}\circ(\pi_F)_{*}=\pi_{*}\circ\widetilde{\rho}_*$为单的,从而$\widetilde{\rho}_*$为单的. 

事实上,%将短正合列$\ref{exact}$,
%看为$\mathcal{M}\longrightarrow M \longrightarrow P$的覆叠变换群之间的关系,则有
考虑下面图表
\begin{diagram}
1&\rTo &\pi_1(M_F)&\rTo^{(\pi_F)_*}&W_F&\rTo^{\psi_F}&\mathbb{Z}_2^{n-1} & \rTo &1\\
&      &\dTo^{\widetilde{\rho}_*}&       &\dTo^{\rho_{*}} &&\dTo&\\
1&\rTo &\pi_1(M) &\rTo^{\pi_*} &W&\rTo^{\psi}   &\mathbb{Z}_2^{n}   & \rTo &1\\
\end{diagram}
其中$\mathbb{Z}_2^{n-1}=\mathbb{Z}_2^{n}/\langle e_1\rangle$.
将$W_F$换成$W'_F$,则$\psi\circ\rho_*(F_k)=\psi\circ\rho_*(F_kF^{-1})=\psi(F_kF^{-1})=\lambda(F_k)/\langle e_1\rangle=\psi_F(F_k)$,所以右边矩形也是交换的,
此时由五引理也可以得到$\widetilde{\rho}_*$单.
 $\hfill{} \Box$
 
由于flag多面体的face也是flag的,所以对任意$k$维面$f$,由归纳知,也有类似的结论.
 {\cor 单多面体$P$为flag,$\rho_*:W_f\longrightarrow W$和$\widetilde{\rho}_*:\pi_1(M_f,p_0)\longrightarrow \pi_1(M,p_0)$为单同态.}
 
 {\defn 
 我们称一个连通闭流形$M$为aspherical的,若$\pi_k(M)=0, k\geq 2$.}
 {\conj   设$f:M\longrightarrow N$为同伦等价,其中$M,N$为同维数闭的aspherical 流形,则$f$同伦于一个同胚映射。}
{\thm [\cite{DJS1}] Let $M$ be a small cover of $P$. Then the following statements are equivalent.\\
%
%设$M$为单多面体$P$上的small cover,则下列条件等价.\\
1、$M$ is aspherical.\\
%2、单多面体$P$的边界dual to 一个flag complex. \\
2、The boundary of $P$ is dual to a flag complex. \\
3、The natural piecewise Euclidean metric on the dual cubical cellulation of $M$ is nonpositively curved.}

{\thm [\cite{F1}] Let $f:N\longrightarrow M$ be a homotopy equivalence between closed smooth manifolds such that $M$ supports a non-positively curved Riemannian metric. Then $N$ and $M$ are stably homeomorphic; i.e.
\begin{equation}\label{eq8}
f\times id :N\times \mathbb{R}^{m+4}\longrightarrow M\times \mathbb{R}^{m+4}
\end{equation}
is homotopic to a homeomorphism where $m=\dim{M}$.}

上面定理说明,当流形$M$是一个non-positively curved Riemannian 流形,且$\dim (M)\neq 3,4$,Borel conjecture 成立. 可以验证 small cover 为这样的闭流形. 
%%%%%%%%%%%%%%%%%%%%
{\cor 设$n(>4)$维闭流形$M,N$都为flag单多面体上的small cover ,若$\pi_1(M)\cong \pi_1(N)$,则$M$和$N$是同胚的.
%$M$是aspherical的.且在这个范畴中,Borel conjecture 成立.
}

对于$3$维情况,在\cite{AFW} 中指出,Borel conjecture 对所有的三维流形都成立. 我们感兴趣的是 aspherical small cover $M$是不是一个Haken manifold
%(irreducible,sufficiently large 3-manifolds 定义见\cite{W1}).

%{\defn 
%A {Haken 3-manifold} is a compact, orientable, irreducible 3-manifold which either

%\item[(1)]  has non-empty boundary which is a collection of incompressible surfaces or

%\item[(2)] is closed, and admits an embedded, closed, orientable, incompressible surface.
%}
{\defn 
A {Haken $3$-manifold} is a compact $3$-manifolds which are

\item[(1)]  $P^2$-irreducible and

\item[(2)] sufficiently large -- i.e. contain a properly embedded, 2-sided, incompressible surface.
}
%{\thm [\cite{F1}] 当三维可定向aspherical闭流形$M$和$N$为Haken manifold时,则Borel 猜想成立.}
{\prop  $P^2$-irreducible M wtih $H_1(M)$ infinite is a Haken manifold.}\\
{\bf 证明:}由Hempel \cite{He1} Lemma 6.6 知$M$包含一个 properly embedded 2-sided, nonseparating incompressible surface $S$. $\hfill{} \Box$
%$M$ 为可定向的,$S$ 2-sided 等价于 $S$ 在$M$中的 normal bundle 是平凡的,等价于$S$是可定向的,Hatcher \cite{Ha1}.   $\hfill{} \Box$
{\lem[*待证] 设$M$是$3$维flag多面体$P$上的一个small cover,则$H_1(M)$ 是infinite的.}\\
{\bf 证明:}
%(有问题不容易解决)由引理\ref{lemm}~知,$\rho_*|_{G_F}=id$,而$G-G_F$与$G_F$是独立的,所以$H_1(M_F)=\pi_1^{\text{ab}}(M_F)$是$H_1(M)=\pi_1^{\text{ab}}(M)$的直和项. 又$P$为flag的,所以二维闭曲面$M_F$不是$\mathbb{R}P^2$,故$\mathbb{Z}$为$H_1(M_F)$的直和项. 所以$H_1(M)$ infinite.
%另一个想法。与F横截相交于点$p_0$的一维面记为$a$,如果$M_F$ 2-sided,则$\pi^{-1}(a)$为$H_1(M)$的 infinite 元. 如果$M_F$ 1-sided, 由flag性知$\mathbb{Z}$为$H_1(M_F)$的直和项,选择对应它的一个生成元$g$,则$g\pi^{-1}(a)$为$H_1(M)$的无限阶元.

{\cor  Aspherical small covers are Haken manifolds.
}\\
{\bf 证明:}设$P$为三维单的flag多面体(不包含单形面),$M$为$P$上的small cover,$M$是aspherical的. 由Sphere Theorem,我们知道三维闭可定向流形是aspherical的,当且仅当它是irreducible 和$\pi_1$ infinite的. 进一步为$P^2$-irreducible,又$H_1(M)$为infinite的,故aspherical small cover $M$ 为Haken 流形. $\hfill{} \Box$

{\thm 在三维small cover 范畴中,下列条件等价.\\
\item[(1)]  $M$是aspherical 的,
\item[(2)]  $M$是$P^2$-irreducible 的.
\item[(3)]  $M$是Haken 的.

进一步,此时$H_1(M)$ 是infinite的.*
}
%设$F$为多面体$P$的任意一个余$1$维面,$M_F$为诱导的small cover,设. 
%从而$M$是irreducible(embedded $S^2$ bounds an embedded $3$-ball). $\rho_{*}:\pi_1(M_F)\longrightarrow \pi_1(M)
%88$为单同态,知$M_F$为$M$中的incompressible曲面,从而$M$为irreducible and sufficiently large的. 从而Borel 猜想成立. $\hfill{} \Box$
%空间$M/M_F$.   $M-M_F$

%对于一个small cover,它的facets对应的$n-1$维对应的面流形中可能不存在(可定向的)incompressible surface,比如正十二面体上的small cover.
%{\exmp 设$P$为一个正十二面体,它的每个面都是一个五边形,根据$4$-color theorem,我们可以仅用$\{e_1,e_2,e_3,e_1e_2e_3\}$去染色,此时决定的small cover是可定向的,而每个五边形面决定的面子流形都是不可定向的.}

可定向 aspherical 流形为irreducible的,故为prime的,故同伦等价诱导同胚. 
尽管$\{M_F\}$中不一定存在$M$的two-sided incompressible surface(考虑正十二面体上的small cover),但$M_F$拥有许多好的性质,比如包含映射诱导的基本群同态为单的.
特别地,我们沿着$M_F$去切$M$,这在我们构造的胞腔结构中,相当于把与$F$横截相交的那些一维闭路$\{x\}$切开,或者给面$F$一个平凡的color. 此时对应在我们上面构造的胞腔结构中,是把facets对应的闭路切开,从而这些闭路决定的生成元变为平凡元,基本群得到化简. 所以从这里我们猜测,类似于Hierachy of Waldhausen 的操作对 (高维)aspherical small cover也是有效的.
%incompressible surface 可以推广为由facets 决定的面子流形.
 这对高维small cover 中的borel 猜想的证明提供一种可行的idea. 而且Davis等一些人已经在做了一些这方面的工作,比如对高维Haken 流形的推广.
\newpage
\begin{thebibliography}{99}
\bibitem{AFW} M. Aschenbrenner, S. Friedl and H. Wilton, 3-manifold groups, {\em Mathematics} (2013), 1-149
\bibitem{Buch} V.M. Buchstaber and T.E. Panov, Torus actions and their applications in topology and combinatorics, {\em University Lecture Series, 24. American Mathematical Society, Providence, RI,} (2002)
\bibitem{D1} M.W. Davis, Groups generated by reflections and aspherical manifolds not covered by Euclidean space, {\em Ann. Math. (2)} 117 (1983), 293-325.
\bibitem{D2} M.W. Davis, Exotic aspherical manifolds, {\em Topology of high-dimensional manifolds.}  (Trieste, 2001), 371-404.
\bibitem{DJ1} M.W. Davis and T. Januszkiewicz, Convex polytopes, coxeter orbifolds and torus actions, {\em Duke Math. J.} 62 (1991), 417-451.
\bibitem{DJS1} M.W. Davis, T. Januszkiewicz, and R.Scott, Nonpositive curvature of blow-ups, {\em Selecta Math.(N.S.)} 4 (1998), 491-547.
\bibitem{DJS2} M.W. Davis, T. Januszkiewicz, and R.Scott, Fundamental groups of blow-ups, {\em Advances in mathematics.} 177 (2003), 115-179.
\bibitem{F1} F.T. Farrell, The Borel conjecture, {\em Topology of high-dimensional manifolds.}  (Trieste, 2001), 225-298.
\bibitem{Ha1} A. Hatcher, Spaces of Incompressible Surfaces, {\em Mathematics.}  (1999).
\bibitem{He1} J. Hempel, 3-manifolds, {\em Annals of Mathematics studies.} 86 (1978).
\bibitem{W1} F. Waldhausen, On irreducible 3-manifolds which are sufficiently large, {\em Ann. of Math.}  (2)87 (1968), 56-88.
\bibitem{KMY1} S. Kuroki, M. Masuda, and L. Yu, Small covers, infra-solvmanifolds and curvature, {\em Forum mathematicum.} 27(5)(2015), 2981-3004
\bibitem{Y1} L. Yu, Crystallographic groups with cubic normal fundamental domain, {\em RIMS Kôkyûroku Bessatsu, B39, Res. Inst. Math. Sci. (RIMS), Kyoto.} (2013), 233-244
\end{thebibliography}

\end{document}